\begin{Solution}{15}
Перейдем в систему отсчета, связанную с колечком $O'$. В этой системе отсчета скорость точки $O$ равна $v_1/ \cos \alpha$ и направлена вверх, так как нить нерастяжима и относительно колечка $O'$ веревка выбирается с постоянной скоростью $v_1$. Поэтому относительно прямой $АА'$, связанной с землей, скорость колечка $O$ будет равна $v_2 = \frac{v_1}{\cos \alpha} - v_1 = v_1 \frac{\sin^2 (\alpha /2)}{\cos \alpha}$ и направлена вверх.
\end{Solution}
\begin{Solution}{16}
К моменту времени $t$ от начала движения клин пройдет расстояние $s = at^2/2$ и приобретет скорость $v_1 = at$. За это время грузик переместится вдоль клина на такое же расстояние $s$, а его скорость относительно клина будет равна $v_2 = at$ и направлена вдоль клина вверх. Скорость грузика относительно земли равна $\vec{v}_3 = \vec{v}_1 + \vec{v}_2$. Поскольку угол между векторами $\vec{v}_1$ и $\vec{v}_2$ равен $\alpha$, то $v_3 = 2at \sin (\alpha /2)$. \\
Угол, который скорость $\vec{v}_3$ составляет с горизонтом, равен $\beta  = (\pi - \alpha)/2$. Таким образом, грузик движется вдоль прямой, составляющей с горизонтом угол $\beta$ c ускорением $a_2 = 2a \sin (\alpha /2)$.
\end{Solution}
\begin{Solution}{23}
Пусть время движения от соударения до возвращения в точку бросания равно $t_2$. Поскольку при упругом ударе вертикальная составляющая скорости не меняется, а горизонтальная скорость увеличивается до величины $v_0 \cos \alpha + 2u$, то полное время полета составит $t_1 + t_2 = \frac{2v_0 \sin \alpha}{g}$. Расстояние по горизонтали от места броска до места удара о стенку выражается в виде $v_0 \cos \alpha t_1 = (v_0 \cos \alpha +2u)t_2$, откуда $t_1 = \frac{v_0 \sin \alpha (v_0 \cos \alpha + 2u)}{g(v_0 \cos \alpha + u)}$.
\end{Solution}
\begin{Solution}{28}
Требование минимальности скорости бросания камня с поверхности земли означает, что оптимальная траектория камня пройдет через точки "вершины" крыши В и С, расположенные на высотах $H$ и $h$ соответственно. Обратим движение камня и определим минимальную скорость в точке C -- $v_C$, при которой камень может перелететь через точку В. Учитывая решение предыдущей задачи, легко понять, что $v_B = \sqrt{g \left(\sqrt{l^2 + (H-h)^2} + H-h \right)}$. Минимально возможную скорость бросания камня с земли $v_{min}$ найдем из закона сохранения энергии $v_{min}^2 = v_B^2 + 2gH$, откуда $v_{min} = \sqrt{g \left(\sqrt{l^2 + (H-h)^2} + H + h \right)}$.
\end{Solution}
\begin{Solution}{105}
Задача 391 сборника задач Сивухина.
\end{Solution}
\begin{Solution}{110}
Решение задачи 408 сборника задач Сивухина.
\end{Solution}
\begin{Solution}{119}
Изменение уровня жидкости на высоте \(z\) описывается дифференциальным уравнением
\[S\left( z \right)\frac{{dz}}{{dt}} = q\left( z \right),\]
где \(S\left( z \right)\) -- площадь поперечного сечения сосуда на высоте \(z,\) а \(q\left( z \right)\) -- поток жидкости, зависящий от высоты $z$.

Принимая во внимание геометрию сосуда, можно предположить, что \[q\left( z \right) =  - \pi {a^2}\sqrt {2gz} ,\] где \(a\) -- радиус отверстия на дне конического сосуда. Учитывая, что отверстие достаточно малое, осевое сечение можно рассматривать как треугольник. Из подобия треугольников следует, что \[\frac{R}{H} = \frac{r}{z}.\] Следовательно, площадь поверхности жидкости на высоте \(z\) будет равна
\[
{S\left( z \right) = \pi {r^2} }
= {\pi {\left( {\frac{{Rz}}{H}} \right)^2} }
= {\frac{{\pi {R^2}{z^2}}}{{{H^2}}}.}
\]
Подставляя \(S\left( z \right)\) и \(q\left( z \right)\) в дифференциальное уравнение, имеем:
\[\frac{{\pi {R^2}{z^2}}}{{{H^2}}}\frac{{dz}}{{dt}} =  - \pi {a^2}\sqrt {2gz} .\]
После простых преобразований получаем следующее дифференциальное уравнение:
\[{z^{\large\frac{3}{2}\normalsize}}dz =  - \frac{{{a^2}{H^2}}}{{{R^2}}}\sqrt {2g} dt.\]
Проинтегрируем обе части, учитывая, что уровень жидкости уменьшается от начального значения \(H\) до нуля за время \(T:\)
\[
{\int\limits_H^0 {{z^{\large\frac{3}{2}\normalsize}}dz}  =  - \int\limits_0^T {\frac{{{a^2}{H^2}}}{{{R^2}}}\sqrt {2g} dt} ,}\;\;
{\Rightarrow \left. {\left( {\frac{{{z^{\large\frac{5}{2}\normalsize}}}}{{\frac{5}{2}}}} \right)} \right|_0^H = \frac{{{a^2}{H^2}}}{{{R^2}}}\sqrt {2g} \left[ {\left. {\left( t \right)} \right|_0^T} \right],}\;\;
{\Rightarrow \frac{2}{5}{H^{\large\frac{5}{2}\normalsize}} = \frac{{{a^2}{H^2}}}{{{R^2}}}\sqrt {2g} T,}\;\;
{\Rightarrow \frac{1}{5}\sqrt {\frac{{2H}}{g}}  = \frac{{{a^2}}}{{{R^2}}}T,}\;\;
{\Rightarrow T = \frac{{{R^2}}}{{5{a^2}}}\sqrt {\frac{{2H}}{g}} .}
\]
\end{Solution}
\begin{Solution}{120}
см. Юмашев Интегралы и производные в физике
\end{Solution}
\begin{Solution}{151}
Поскольку система зарядов замкнута, то полная энергия и импульс системы будет сохраняться, отсюда следует следующая система уравнений:
\begin{equation*}
\begin{cases}
\frac{q^2}{L}=\frac{q^2}{r}+\frac{m V^2}{2}+\frac{M U^2}{2} \\
m \cdot U=M \cdot V
\end{cases}
\end{equation*}
,где $r$-расстояние между частицами, $V$ и $U$-скорость первой и второй частицы на расстоянии $r$ \\
Решая эту систему уравнений получим: $U=\sqrt{\frac{2 q^2 m (\frac{1}{L}-\frac{1}{r})}{(M+m) M}}$ и $V=\sqrt{\frac{2 q^2 M (\frac{1}{L}-\frac{1}{r})}{(M+m)m}}$. Для получения ответов устремляем $r$ к бесконечности, приравниваем к $7L$
\end{Solution}
\begin{Solution}{152}
По теореме взаимности Грина: $Q_1\varphi_1+q\varphi+Q_2\varphi_2=0$, где $\varphi_1$, $\varphi_2$ и $\varphi$-потенциалы на поверхности малой, большой сферы и между ними соответственно, если поставить в геометрический центр сфер заряд($\varphi_1=\frac{q`}{a}$, $\varphi_2=\frac{q`}{b}$, $\varphi=\frac{q`}{r}$), $q`$, $Q_1$, $Q_2$ и $q$-заряды на поверхности малой, большой сферы и между ними соответственно(при отсутствии других зарядов). Так же система замкнута, поэтому действует закон сохранения заряда: $Q_1+Q_2+q=0$.\\
Решая эти уравнения относительно $Q_1$ и $Q_2$ можно получить: $Q_1=-q\frac{(b-r)a}{(b-a)r}$, $Q_2=-q\frac{(r-a)b}{(b-a)r}$
\end{Solution}
\begin{Solution}{153}
По теореме взаимности Грина: $Q_1\varphi_1+q\varphi+Q_2\varphi_2=0$, где $\varphi_1$, $\varphi$ и $\varphi$
\end{Solution}
\begin{Solution}{197}
На частицу действуют сила вязкого трения $\vec{F} = - r\vec{v}$ и сила Лоренца $\vec{F_L} = q \vec{v} \times \vec{B}$, уравнение движения в проекциях на оси, перпендикулярные $\vec{B}$, имеет вид:  $m \dot{v_x}=-rv_x+q v_y B$, $m \dot{v_y}=-rv_y-q v_x B$. Умножим второе из этих уравнений на $i$ и сложим с первым, получим $m\left( \dot{v_x} +i \dot{v_y} \right) = -r(v_x+i v_y) - iqB(v_x+i v_y)$. Обозначим $\phi \equiv v_x+i v_y$, тогда получим для $\phi$ дифференциальное уравнение $\dot{\phi} + \delta \phi + i\omega \phi=0$, где $\delta=r/m$,  $\omega=qB/m$. Решение этого уравнения, удовлетворяющее начальным условиям $v_x(0)=v_0$, $v_y(0)=0$, имеет вид $\phi(t)=v_0 e^{-(\delta +i\omega) t}$. Расстояние от начальной точки до точки, где частица остановится, $L=\sqrt{x_{\infty}^2 + y_{\infty}^2} =\sqrt{ \left( \int_{0}^{\infty}{v_x \, dt} \right)^2 + \left( \int_{0}^{\infty}{v_y \, dt} \right)^2}=\vert \int_{0}^{\infty}{ \phi \, dt} \vert=v_0/\sqrt{\delta^2 +\omega^2}=m v_0/\sqrt{r^2+q^2B^2}$.
\end{Solution}
