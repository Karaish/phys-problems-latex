\begin{sol}
а) Кольцо закреплено. Частица будет отталкиваться от кольца, её скорость будет уменьшаться. Для того, чтобы частица достигла центра кольца, нужно, чтобы величина её скорости не успела упасть до нуля. Для нахождения минимальной начальной скорости частицы, следует принять её скорость в центре кольца равной нулю. Согласно закону сохранения энергии:
\begin{equation}\label{eq}
W_1 + \frac{m V^2}{2} = W_2
\end{equation}
Потенциальная энергия взаимодействия точечных зарядов: $W = \frac{k q_1 q_2}{r}$ \\
Выведем выражение для потенциальной энергии взаимодействия для частицы и кольца. Разобьем кольцо на бесконечно малые элементы. В любой момент времени расстояние $r$ от частицы до любого элемента кольца будет одиннаковым. Тогда потенциальная энергия взаимодействия Кольца и i-ого элемента: $\Delta W = \frac{k q \Delta q_i}{r}$ \\
Суммарная энергия взаимодействия кольца и частицы: $W = \sum \frac{k q \Delta q_i}{r} = \frac{k q}{r} \sum \Delta q_i = \frac{k q^2}{r}$ \\
Начальная потенциальная энергия (при $r \rightarrow \infty$): $W = 0$. Потенциальная энергия в момент, когда частица окажется в центре кольца ($r = R$): $W = \frac{k q^2}{R}$. Тогда из \ref{eq} получаем:
\begin{equation*}
\frac{m V^2}{2} = \frac{k q^2}{R} \qquad \rightarrow \qquad V_{min} = q \sqrt{\frac{2 k}{m R}}
\end{equation*}
б) Если кольцо свободное, то отталкиваясь от заряда, оно тоже придет в движение. Для того, чтобы в этом случае частица достигла центра кольца, величина скорости частицы к этому моменту все ещё должна превышать скорость кольца. Для нахождения минимальной начальной скорости частицы следует считать, что скорость в тот момент, когда она достила центра кольца, скорости частицы и кольца сравнялись. Пусть в момент, когда частица достигла центра кольца, их скорости будут равны $U$. Запишем законы сохранения импульса и энергии для замкнутой системы тел и определим начальную минимальную скорость частицы:
\begin{equation*}
\begin{cases}
mV = m U + 3 m U \\
\frac{m V^2}{2} = \frac{k q^2}{R} + \frac{m U^2}{2} + \frac{3 m U^2}{2} \qquad \rightarrow \qquad V_{min} = q \sqrt{\frac{8 k}{3 m R}}
\end{cases}
\end{equation*} 
\end{sol}