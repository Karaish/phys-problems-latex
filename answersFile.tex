\begin{Answer}{1}
$\frac{L}{2v}$
\end{Answer}
\begin{Answer}{2}
$\frac{uL}{v}$
\end{Answer}
\begin{Answer}{3}
10 мин
\end{Answer}
\begin{Answer}{4}
$v_p = (S - S_1)/t$
\end{Answer}
\begin{Answer}{5}
$u > v / \tan \alpha$
\end{Answer}
\begin{Answer}{6}
Под углом $\alpha$ к вертикали, $\tan \alpha = \omega r / v_0$
\end{Answer}
\begin{Answer}{7}
$w = \sqrt{v^2 + u^2 - 2uv\cos \alpha_0}$,
$\alpha = \alpha_0 + \arcsin \left( \frac{u}{w} \sin \alpha \right)$
\end{Answer}
\begin{Answer}{8}
$\vec{v} = \vec{v}_1 + \vec{v}_3 - \vec{v}_2$,
$v = \sqrt{v_1^2 + v_2^2 + v_3^2}$,
\end{Answer}
\begin{Answer}{9}
$\cos \alpha = v/u, u > v$, либо $\cos \alpha = u/v, u < v$
\end{Answer}
\begin{Answer}{10}
$u + 2v$
\end{Answer}
\begin{Answer}{11}
$L/\sqrt{2}$
\end{Answer}
\begin{Answer}{12}
$g \tau t + g \tau^2 / 2$
\end{Answer}
\begin{Answer}{13}
$L_{\max} = \frac{(v_1 + v_2)^2}{2(a_1+a_2)}$
\end{Answer}
\begin{Answer}{14}
$v^2 = \pi g R n$, где $n \in \mathbb{N}$
\end{Answer}
\begin{Answer}{15}
$v_2 = v_1 \frac{\sin^2 (\alpha /2)}{\cos \alpha}$
\end{Answer}
\begin{Answer}{16}
Ускорение грузика относительно земли $a_2 = 2a \sin (\alpha /2)$ направлено под углом $\beta  = (\pi - \alpha)/2$ к горизонту.
\end{Answer}
\begin{Answer}{17}
При $v_a > vL/h$ человек ни при каком угле не сможет оказаться на шоссе раньше автобуса. При $v_a < vL/h$ существует два ответа $\alpha_1 = \arcsin \frac{h}{L} + \arcsin \frac{v_ah}{vL}$ и $\alpha_2 =  \arcsin \frac{v_ah}{vL} - \arcsin \frac{h}{L}$.
\end{Answer}
\begin{Answer}{18}
$l/2 - vt \cos \alpha$
\end{Answer}
\begin{Answer}{19}
$\tan \beta = \sqrt{\tan^2 \alpha + \frac{2gh}{v_0^2 \cos^2 \alpha}}$, $s=\frac{v_0 \cos \alpha}{g} \sqrt{v_0^2 \sin^2 \alpha + 2gh}$.
\end{Answer}
\begin{Answer}{20}
$h = g\tau^2/2 = 5$ м		
\end{Answer}
\begin{Answer}{21}
$h = gt^2/2 = 5$ м
\end{Answer}
\begin{Answer}{22}
$h = gt_1t_2/2$
\end{Answer}
\begin{Answer}{23}
$t_1 = \frac{v_0 \sin \alpha (v_0 \cos \alpha + 2u)}{g(v_0 \cos \alpha + u)}$
\end{Answer}
\begin{Answer}{24}
$v_{min} = \sqrt{g(R+2H)}$
\end{Answer}
\begin{Answer}{25}
$y = \frac{v_0^2}{2g} - \frac{gx^2}{2v_0^2}$
\end{Answer}
\begin{Answer}{26}
Если $v_0 < 2 \sqrt{gh}$, то дальность полета $L = \frac{v_0^2}{g}$; если $v_0 > 2 \sqrt{gh}$, то $L = 4h \sqrt{\frac{v_0^2}{2gh}-1}$.
\end{Answer}
\begin{Answer}{27}
$v_{min} = \sqrt{g \left(\sqrt{l^2 + h^2} + h \right)}$
\end{Answer}
\begin{Answer}{28}
$v_{min} = \sqrt{g \left(\sqrt{l^2 + (H-h)^2} + H + h \right)}$
\end{Answer}
\begin{Answer}{29}
1) $\pi/2$; 2) $\pi/8$
\end{Answer}
\begin{Answer}{30}
1) При $gR > v^2$ максимальная высота подъема $h = 2R$, при $gR < v^2$ -- $h = R +\frac{v^2}{2g} + \frac{gR^2}{2v^2}$; 2) $v_{min} = \frac{\pi g R}{(1+ \sin \alpha) \cos \alpha}$, где $\sin \alpha = gR/v^2$; 3) изменится.
\end{Answer}
\begin{Answer}{31}
$R = \sqrt{g} T_1 T_2 /2\sqrt{2}$
\end{Answer}
\begin{Answer}{32}
$s = 10$ м, $\tau = v_0 \sin 150^{\circ} \cos 30^{\circ} \cos 45^{\circ} /g$
\end{Answer}
\begin{Answer}{33}
$\tau = \frac{v_0 \cos (\alpha /2)}{g \sin (3\alpha / 2)}$
\end{Answer}
\begin{Answer}{34}
$x = v \sqrt{2h/g} \left( 1 + \alpha \right) / \left( 1 - \alpha \right)$
\end{Answer}
\begin{Answer}{35}
$v(\varphi) = v\sqrt{2(1- \cos \varphi)}$, $x(t) = vt - R \sin \varphi$, $y(t) = R(1 - \cos \varphi)$, $\varphi = vt / R$, $R_k = 4R$.
\end{Answer}
\begin{Answer}{36}
$u = v \cos \alpha / \cos \beta$
\end{Answer}
\begin{Answer}{37}
$T_1 = 2 \pi L/v$, $T_2 = 2 \pi L/3v$.
\end{Answer}
\begin{Answer}{38}
$\omega = v_0/l_0 \sin \alpha$, $R = sqrt{l^2 + l_0^2 - 2ll_0 \sin \alpha}$, $v_M = \omega R$.
\end{Answer}
\begin{Answer}{39}
$a_p = \sqrt{a^2 + a^4t^4/R^2}$
\end{Answer}
\begin{Answer}{40}
$v = \omega l \tan \alpha$
\end{Answer}
\begin{Answer}{41}
$v = \omega l/ \cos^2 \varphi$
\end{Answer}
\begin{Answer}{42}
$\omega = u \sin^2 \alpha /h$
\end{Answer}
\begin{Answer}{43}
$a = uv/L$
\end{Answer}
\begin{Answer}{44}
$\omega R/ \cos (\alpha /2)$
\end{Answer}
\begin{Answer}{45}
отрезок
\end{Answer}
\begin{Answer}{46}
$4 \omega^2 R$
\end{Answer}
\begin{Answer}{47}
$u = \sqrt{v^2 + \omega^2 L^2}$, $a = \sqrt{\omega^4 L^2 + 4\omega^2 v^2}$
\end{Answer}
\begin{Answer}{48}
В центре через $t = v/l$, $\tan \alpha = (x-y)/(x+y)$.
\end{Answer}
\begin{Answer}{49}
если $\mu > \tan \alpha$, то $a = 0$; если $\mu < \tan \alpha$, то $a = g(\sin \alpha - \mu \cos \alpha)$.
\end{Answer}
\begin{Answer}{50}
$a_1 = g \frac{m \sin^2 \alpha}{m \sin^2 \alpha + M \cos^2 \alpha}$, $a_2 = g\frac{m \sin \alpha \cos \alpha}{m \sin^2 \alpha + M \cos^2 \alpha}$
\end{Answer}
\begin{Answer}{51}
$a_1 = g \frac{m_2 \sin \alpha \cos \alpha}{m_1 + m_2 \sin^2 \alpha}$, $t=\sqrt{\frac{2h(m_1+m_2\sin^2\alpha)}{(m_1+m_2)g \sin^2 \alpha}}$
\end{Answer}
\begin{Answer}{52}
$v_{min} = u \sin \alpha = 5$ м/c, $t = v_2/g(\sin \alpha - \mu \cos \alpha) = 11,3$ с, парабола.
\end{Answer}
\begin{Answer}{53}
$\tan \alpha = \mu$, $F = \mu mg / \sqrt{\mu^2 + 1}$
\end{Answer}
\begin{Answer}{54}
$\tan \beta = \mu$, $F = mg \sin (\alpha + \beta)$ при $\alpha + \ beta < \pi/2$, иначе $F = mg$.
\end{Answer}
\begin{Answer}{55}
$F_2 = mg \sqrt{(\mu \cos \alpha)^2 - (\sin \alpha)^2} = 48$ Н, $F_3 = mg(\mu \cos \alpha + \sin \alpha)/\sqrt{\mu^2 + 1} = 93,1$ Н.
\end{Answer}
\begin{Answer}{56}
$F = \mu m_1 g \left(m_1/m_2 + 1 \right)$
\end{Answer}
\begin{Answer}{57}
$F = 20100 \mu mg = 241,2$ Н
\end{Answer}
\begin{Answer}{58}
$\cos \alpha = g/\omega^2 R$, при $g < \omega^2 R$, иначе $\alpha = 0$.
\end{Answer}
\begin{Answer}{59}
$\nu = \frac{1}{2\pi}\sqrt{\frac{g}{L \cos \alpha}}$
\end{Answer}
\begin{Answer}{60}
При $\varepsilon < \mu g /R$ $t=\frac{1}{\sqrt{\varepsilon}}\left( \frac{\mu^2 g^2}{\varepsilon^2 R^2} - 1 \right)^{\frac{1}{4}}$, $\tan \alpha = \left( \frac{\mu^2 g^2}{\varepsilon^2 R^2} - 1 \right)^{-\frac{1}{24}}$
\end{Answer}
\begin{Answer}{61}
$x = \frac{2F}{4\pi^2 \nu^2 m} = 25,3$ см, $x = v\sqrt{\frac{m}{k}}=15,9$ мм
\end{Answer}
\begin{Answer}{62}
$T(x) = m \omega^2 \left( l^2 - 4x^2 \right)/8l$
\end{Answer}
\begin{Answer}{63}
$T = 2 \pi \sqrt{\frac{L^3}{G(m_1+m_2)}}$
\end{Answer}
\begin{Answer}{64}
В первом случае $T = 9mv^2 / 4l$, в кинетическую энергию вращения перейдет $3/4$ энергии
\end{Answer}
\begin{Answer}{65}
$T = \frac{\pi}{2}\sqrt{\frac{m}{2k}}$
\end{Answer}
\begin{Answer}{66}
$v_1 = v/3$, $v_2 = 2v/3$
\end{Answer}
\begin{Answer}{67}
$4m_1m_2/(m_1+m_2)^2$
\end{Answer}
\begin{Answer}{69}
Окружности радиусами $r_1 = mR/(m+M)$ и $r_2 = MR/(m+M)$
\end{Answer}
\begin{Answer}{70}
$x = 3$ см
\end{Answer}
\begin{Answer}{71}
$F = \rho S v^2$
\end{Answer}
\begin{Answer}{72}
$\vec a = - \mu \vec u / M$
\end{Answer}
\begin{Answer}{73}
$T = m\omega^2R/2 \pi$
\end{Answer}
\begin{Answer}{74}
$e^{2\pi \mu n}$
\end{Answer}
\begin{Answer}{75}
$F = 3mg(1-x/l)$
\end{Answer}
\begin{Answer}{76}
$v = \sqrt{gH}$
\end{Answer}
\begin{Answer}{77}
$F = mgH/l$, $a = gH/l$
\end{Answer}
\begin{Answer}{78}
$h = l(1 - \sin \alpha)/2 \cos \alpha$
\end{Answer}
\begin{Answer}{79}
$y = a \cosh x/a$, $a = T/\rho g H$
\end{Answer}
\begin{Answer}{80}
$v > \sqrt{\mu g l(1+m/M)/2}$
\end{Answer}
\begin{Answer}{81}
$L = hM/\mu(m+M)$
\end{Answer}
\begin{Answer}{82}
$x_m = 2(F-\mu mg)/k$, $v_m = \sqrt{\frac{k}{m}}\frac{F-\mu mg}{k}$
\end{Answer}
\begin{Answer}{83}
$A=2mgh - mv^2/2$
\end{Answer}
\begin{Answer}{84}
$v^2 = (v_1^2 + v_2^2)/2 - g(H-h)$
\end{Answer}
\begin{Answer}{85}
$v = \sqrt{2gH(1+m/M)}$
\end{Answer}
\begin{Answer}{86}
$v_1 = Mv/(M+2m) > v_2 = v(M-m)/(M+m)$
\end{Answer}
\begin{Answer}{87}
$v_0 = \sqrt{\frac{gLM}{(M-m) \sin 2\alpha}}$
\end{Answer}
\begin{Answer}{88}
$N = s(M^2-m^2)/(Lm^2)$
\end{Answer}
\begin{Answer}{89}
$v_{1,2} = (2\pm \sqrt{3})/4$
\end{Answer}
\begin{Answer}{90}
$v = \frac{2}{3}\sqrt{\frac{2gl}{3}}$
\end{Answer}
\begin{Answer}{91}
$v = \sqrt{24gl/5}$
\end{Answer}
\begin{Answer}{92}
$v_1 = \sqrt{2FL/m}=27,4$ м/с, $v_2 = \sqrt{6FL/(3m+M)} = 20,2$ м/с
\end{Answer}
\begin{Answer}{93}
$Q = N/q \rho V = 26,5$ л/ч,  $v=\sqrt[3]{\frac{NRT}{2pS\mu}}=33$ м/с
\end{Answer}
\begin{Answer}{94}
$h= \frac{v_0^2 M}{2g(m+M)}$, $v_2 = 2mv_0/(m+M)$, $T=m(g+v_0^2/l)$
\end{Answer}
\begin{Answer}{95}
$a=g(m_1-m_2)/(m_1 +m_2 + I/r^2)$
\end{Answer}
\begin{Answer}{96}
$a = F(\cos \alpha - r/R)/(m+I/R^2)$
\end{Answer}
\begin{Answer}{97}
$T=2mg$
\end{Answer}
\begin{Answer}{98}
$t=mr^2\omega/2M_0$, $N=M_0t^2/2I$, $I = I_0+m(d^2+r^2/2)$
\end{Answer}
\begin{Answer}{99}
$l=2L/3$
\end{Answer}
\begin{Answer}{100}
$N=3 \pi R \nu^2/(4\mu g)$
\end{Answer}
\begin{Answer}{101}
$N = mg/4$
\end{Answer}
\begin{Answer}{102}
$N=\sqrt{10}Mg/4$
\end{Answer}
\begin{Answer}{103}
$v=\sqrt{gR(7\cos \alpha - 4)/3}$
\end{Answer}
\begin{Answer}{104}
$\omega_{0 \min} = v_0/R$, $v= (\omega_0R - v_0)/2$
\end{Answer}
\begin{Answer}{105}
Движение после перехода границы будет сначала равнозамедленное, затем с постоянной скоростью; $1/3$ энергии превратится в тепло, $2/9$ во вращательную энергию и $4/9$ останется в виде поступательной энергии движения.
\end{Answer}
\begin{Answer}{106}
$u = \frac{5mv}{7M}\left( 1 - l/R \right)$
\end{Answer}
\begin{Answer}{107}
$\Delta L = -4I_1I_2\omega_0/(I_1+I_2)$, $\Delta E = -2I_1I_2\omega_0^2/(I_1+I_2)$
\end{Answer}
\begin{Answer}{108}
$l = L/\sqrt{3}$
\end{Answer}
\begin{Answer}{109}
$M=2l/\sqrt{3}$, $T = 2\pi \sqrt{l/g}$
\end{Answer}
\begin{Answer}{110}
$\omega = \frac{12mv_0}{(4m+M)l}$
\end{Answer}
\begin{Answer}{111}
1) $l = L/\sqrt{3}$; 2) $u_1 = u_0/3$, $V=u_0/3$, $\omega = u_0/l$.
\end{Answer}
\begin{Answer}{112}
1) верхний удар $x > 2R/5$; 2) нормальный удар $x = 2R/5$; 3) нижний удар $x < 2R/5$.
\end{Answer}
\begin{Answer}{113}
$x(t) = \frac{mv_0}{\alpha}\left( 1 - e^{-\alpha t /m}\right)$, $s = mv_0/\alpha$
\end{Answer}
\begin{Answer}{114}
$v(t) = \sqrt{\frac{mg}{\alpha}} \tanh (kt/m)$, $s = \frac{m}{\alpha} \sqrt{\frac{mg}{\alpha}} \cosh^2 (kt/m)$
\end{Answer}
\begin{Answer}{115}
$\Delta h = \frac{1}{2\beta} \log \frac{e^{2\beta h}}{2 - e^{-2\beta h}}$, $\beta = k/m$
\end{Answer}
\begin{Answer}{116}
$T = \sqrt{\frac{L}{(1+\mu) g}} \arccosh \frac{L}{\varepsilon (1+\mu)g} = \sqrt{\frac{L}{(1+\mu) g}} \log \left[ \frac{L}{\varepsilon (1+\mu) g} - \sqrt{\frac{L^2}{\varepsilon^2 (1+\mu)^2 g^2} - 1}\right]$
\end{Answer}
\begin{Answer}{117}
$h = h_{\max}(1-e^{-S_0vt/S^*h_{\max}})$
\end{Answer}
\begin{Answer}{118}
$h(t) = \left(\sqrt{h_0} - s t \sqrt{g/2} /S \right)^2$
\end{Answer}
\begin{Answer}{119}
$z^{\frac{3}{2}}zt = - \frac{a^2H^2}{R^2}\sqrt{2g}dt$, $T = \frac{R^2}{5a^2}\sqrt{\frac{2H}{g}}$
\end{Answer}
\begin{Answer}{120}
$\tau = L_0(e^{u/v} - 1)/u$
\end{Answer}
\begin{Answer}{121}
$v=v_0/(1+\cos \varphi)$
\end{Answer}
\begin{Answer}{122}
$\tau = \frac{kv_0}{g\sin\alpha(k^2-1)}$, $s = \frac{2v_0^2k^2}{4kg\sin\alpha(k^2-1)}$
\end{Answer}
\begin{Answer}{123}
$L = mv_0/k$, $\tau = kH/mg + m/k$
\end{Answer}
\begin{Answer}{124}
$v_0^2 = \frac{2gR}{4\mu^2+1}\left( 3\mu e^{\mu \pi} - 2\mu^2 +1\right)$
\end{Answer}
\begin{Answer}{125}
$v=gt/4+g\gamma t (\beta^2t^2+3\beta t \gamma+3\gamma^2)/4(\beta t + \gamma)$, $\beta = 4\pi \alpha(3m/4\pi\rho)^{2/3}/3$, $\gamma = m_0^{1/3}$
\end{Answer}
\begin{Answer}{126}
$x=\frac{h}{2(1+u/v)}(\frac{y}{h})^{1+u/v} - \frac{h}{2(1-u/v)}(\frac{y}{h})^{1-u/v} - \frac{h}{2(1+u/v)}+\frac{h}{2(1-u/v)}$, $x_0=h\frac{u}{v}\left(1-\frac{u^2}{v^2}\right)^{-1}$
\end{Answer}
\begin{Answer}{127}
$T_{\max}=p_0V_0/(4R)$
\end{Answer}
\begin{Answer}{128}
$T_{\max} = \sqrt{p_0/3\alpha}$
\end{Answer}
\begin{Answer}{129}
$p_0 = \rho g h \left( (L-h)^2 - 4x^2 \right)/ (4x(L-h))$
\end{Answer}
\begin{Answer}{130}
$t_1 = \tau V(p_n-p)/(p_0V_0)$, $t_2 = \tau \log p/p_n /\log (1+V_0/V)$
\end{Answer}
\begin{Answer}{131}
$h_1 = h \frac{2p_0+3\rho gh}{2(2p_0 + \rho gh)}$, $h_2 = h/2 + p_0/\rho g$
\end{Answer}
\begin{Answer}{132}
$x = a + \sqrt{a^2+1}$, $a = \frac{T (n^2 - 1)}{T_2 2n}$
\end{Answer}
\begin{Answer}{133}
$\Delta M = 2M/15$
\end{Answer}
\begin{Answer}{134}
$T_0 = \sqrt{2}T$, $p_0=p(1+\sqrt{2})2^{-5/4}$
\end{Answer}
\begin{Answer}{135}
$u \approx \frac{\lambda}{2}\sqrt{\frac{k}{3mT}}\frac{dT}{dx}$
\end{Answer}
\begin{Answer}{136}
$u = \frac{\sqrt{T_2}-\sqrt{T_1}}{\sqrt{T_2}+\sqrt{T_1}+4\sqrt{T}} \sqrt{\frac{3RT}{\mu}}$
\end{Answer}
\begin{Answer}{137}
$\mid \frac{dT}{dz} \mid < \frac{2\mu g}{7R}$
\end{Answer}
\begin{Answer}{138}
$T=T_0+2mv^2/3R$
\end{Answer}
\begin{Answer}{139}
$v = v_0\left( 1 + \sqrt{1+9pV/2mv_0^2} \right)/3$
\end{Answer}
\begin{Answer}{140}
$T=T_0 \left( (\eta + 1)^2/4 \eta \right)^{(\gamma-1)/2}$
\end{Answer}
\begin{Answer}{141}
$\eta = \frac{2(T_2-T_1)}{3T_1+5T_2}$
\end{Answer}
\begin{Answer}{142}
$T=2T_0/3 = 200$ К
\end{Answer}
\begin{Answer}{143}
$\Delta h = v_0^2 / 5g = 8$ см
\end{Answer}
\begin{Answer}{144}
1) $U/W=5/2$; 2) $Q=\frac{7\varepsilon_0 SU^2}{4d}$.
\end{Answer}
\begin{Answer}{145}
$A= R(T_2-T_1)-RT_1 \log T_2/T_1$
\end{Answer}
\begin{Answer}{146}
1) $T_0 = m_1gh(1+S_2/S_1)/(\nu R) = 289$ К;
2) $T_2 = T_0 + 2Q/(5\nu R) = 337$ К;
3) $T_3 = \frac{2}{5\nu R}(m_1 gh + m_2 gh)+ \frac{3}{5}T_0 = 250$ К.
\end{Answer}
\begin{Answer}{147}
$C = C_V + R/(1+\alpha)$
\end{Answer}
\begin{Answer}{148}
$ V T^{1/(\gamma - 1)} = C e^{\alpha T^2 / 2R}$
\end{Answer}
\begin{Answer}{149}
$T_2/T_1 = (\alpha i)/(\alpha (1+i) - 1)$, $p_2/p_1 = i/(\alpha (1+i) - 1)$
\end{Answer}
\begin{Answer}{150}
$C(V) = R/(\gamma-1)+R(V_0-2V)/(V_0-2V)$
\end{Answer}
\begin{Answer}{151}
$v = \sqrt{\frac{2Mq^2}{Lm(m+M)}}$, $u = \sqrt{\frac{2mq^2}{LM(m+M)}}$
\end{Answer}
\begin{Answer}{152}
$q_a = -q\frac{a(b-r)}{r(b-a)}$, $q_b = -q\frac{b(r-a)}{r(b-a)}$
\end{Answer}
\begin{Answer}{153}
$F = \frac{Q^2 \pi (R^2-h^2)}{2 \varepsilon_0 {(4 \pi R^2)}^2}$
\end{Answer}
\begin{Answer}{154}
$F = \frac{\sigma^2 \sqrt{3}a^2}{8\varepsilon_0}$
\end{Answer}
\begin{Answer}{155}
$\mid\Delta q \mid = \mid q \frac{x-x_1}{d}$
\end{Answer}
\begin{Answer}{156}
$v_{\min} = q\sqrt{\frac{2k}{mR}}$, $v_{\min 2} = q\sqrt{\frac{8k}{3mR}}$,
\end{Answer}
\begin{Answer}{157}
$I_{01} = \frac{3 \mathcal{E}}{3r+2R}$, $I_{02} = \frac{\mathcal{E}}{3r+2R}$
\end{Answer}
\begin{Answer}{158}
$Q=\frac{C_1C_2 \mathcal{E}^2}{2(C_1+C_2)}$
\end{Answer}
\begin{Answer}{159}
$I = \frac{U}{R} e^{-t/RC}$
\end{Answer}
\begin{Answer}{160}
1) $I_1=\frac{\mathcal{E}_1 + \mathcal{E}_2}{3R}$; 2) $Q=(C_1 \mathcal{E}_1^2 + C_2 \mathcal{E}_2^2/2)$
\end{Answer}
\begin{Answer}{161}
1) $U_1 = \frac{2R \mathcal{E}}{r+3R}$, $U_2 = \frac{R \mathcal{E}}{r+3R}$; 2) $Q=\frac{3RC\mathcal{E}}{r+3R}$
\end{Answer}
\begin{Answer}{162}
$R/3$, $5R/6$, $6R/7$, $7R/15$
\end{Answer}
\begin{Answer}{163}
$I = 2/21$ А
\end{Answer}
\begin{Answer}{164}
$R = \rho r /16d^2$
\end{Answer}
\begin{Answer}{165}
$(6-\sqrt{3})R/6$
\end{Answer}
\begin{Answer}{166}
$R$
\end{Answer}
\begin{Answer}{167}
$R_1 = 2R/n$
\end{Answer}
\begin{Answer}{168}
$t_1 = 6$ мин, $t_2 = 25$ мин
\end{Answer}
\begin{Answer}{169}
$I = 0,06$ А
\end{Answer}
\begin{Answer}{170}
$R = 5$ Ом
\end{Answer}
\begin{Answer}{171}
$U_0 = 2R^2/\alpha$
\end{Answer}
\begin{Answer}{172}
636 В
\end{Answer}
\begin{Answer}{173}
1) $m=\mathcal{E}BL/Rg - M$; 2) $v = \frac{MgR}{B^2L^2}$
\end{Answer}
\begin{Answer}{174}
$\sin \frac {\alpha}{2} = \frac{CUlB}{2m \sqrt{gL}}$
\end{Answer}
\begin{Answer}{175}
$\omega^2 = \frac{2 T_0}{ml} - \frac{\mu \mu_0 I l_1}{\pi a^2}$, $I^* = \frac{2T_0 \pi a^2}{\mu \mu_0 l l_1}$
\end{Answer}
\begin{Answer}{176}
1) $I=\pi v B r^2/R = 251$ мА; 2) $P = I^2 R = 31,6$ мВт; 3) $M = P/2 \pi \nu = 126$ мкН м
\end{Answer}
\begin{Answer}{177}
$a=g/(1+ \varepsilon_0 \pi R^2 d B^2 / m)$
\end{Answer}
\begin{Answer}{178}
$B = \frac{\mu_0 I}{\pi^2 R}$
\end{Answer}
\begin{Answer}{179}
$F = \frac{\mu_0 I^2}{\pi l}$
\end{Answer}
\begin{Answer}{180}
$\omega = qB/2m$
\end{Answer}
\begin{Answer}{181}
$\omega(t) = qB(t)/2m$
\end{Answer}
\begin{Answer}{182}
$v = \frac{Rr}{lB(R+r)}\left( \frac{mg}{Bl} - \frac{\mathcal{E}}{r} \right)$
\end{Answer}
\begin{Answer}{183}

\end{Answer}
\begin{Answer}{184}
$T = 2\pi \sqrt{\frac{l}{g} \left( 1 + \frac{B^2L^2C}{ml} \right)}$
\end{Answer}
\begin{Answer}{185}
$t=\sqrt{\frac{2l}{g \sin \alpha} \left( 1 + \frac{C l^2B^2 \cos^2 \alpha}{m} \right)}$
\end{Answer}
\begin{Answer}{186}
$F=\frac{\mu_0^2 I^2 v}{4 \pi^2 R} \log^2 \frac{b}{a}$
\end{Answer}
\begin{Answer}{187}
$p=\frac{\mu_0^2ab^2I^2}{8\pi^2RL(L+a)}\log\left( 1+\frac{a}{L} \right)$
\end{Answer}
\begin{Answer}{188}
$T_0 = 2 \pi \sqrt{\frac{mL(\rho L + R)}{\mathcal{E}lB}}$
\end{Answer}
\begin{Answer}{189}
1) $Q = 3mv_0^2/8$; 2) $a=\sqrt[3]{\frac{mv_0R}{4B^2}}$; 3) $a=\sqrt{\frac{mv_0R}{4B^2d}}$
\end{Answer}
\begin{Answer}{190}
$d = 2 \pi v \cos \alpha m / qB$
\end{Answer}
\begin{Answer}{191}
$t=\frac{2m}{eB}\arctan \left( \frac{eBR}{mv}\right)$
\end{Answer}
\begin{Answer}{192}
1) $v_0 < eBd/m$; 2) $U < eB^2d^2/2m$
\end{Answer}
\begin{Answer}{193}
$t = \frac{2\pi m_1 m_2}{qB(m_1+m_2)}$
\end{Answer}
\begin{Answer}{194}
$L=2\pi m u n/ eB$, $n \in \mathcal{N}$
\end{Answer}
\begin{Answer}{195}
$\vec{a}_c = \frac{q_1}{m_1}\vec{v}_c \times \vec B$,
${\vec{r}}^{\prime \prime} = \frac{q_1}{m_1}\vec{r}^{\prime} \times \vec B + \frac{q_1(q_2-q_1)\vec r}{m_1 r^3}$
\end{Answer}
\begin{Answer}{196}
$\vec r \times \vec p - \alpha q \vec r/ r = \text{const} $
\end{Answer}
\begin{Answer}{197}
$r_{\infty} = mv_0/\sqrt{\gamma^2 + q^2B^2}$
\end{Answer}
\begin{Answer}{198}
$\langle v \rangle = \frac{E}{B}\left(\frac{\beta}{2\omega} + 2\pi \frac{\beta^2}{\omega^2} \right)$, $\beta = 2\gamma/m$, $\omega = \sqrt{\frac{q^2B^2+\gamma^2}{m^2}} \approx \frac{qB}{m}$
\end{Answer}
\begin{Answer}{199}
$v_0 = 2qB_0/(mk)$
\end{Answer}
\begin{Answer}{200}
$B=\frac{b}{b^2-a^2}\sqrt{\frac{8mU}{e}}$
\end{Answer}
\begin{Answer}{201}
$\omega = \sqrt{F/ml}$
\end{Answer}
\begin{Answer}{202}
$\omega = \sqrt{\frac{F(r+l)}{rml}}$
\end{Answer}
\begin{Answer}{203}
$\omega = \cos \beta \sqrt{2k/m}$
\end{Answer}
\begin{Answer}{204}
$\omega = \sqrt{3g/2l+3k/m}$
\end{Answer}
\begin{Answer}{205}
$T=2\pi \sqrt{\frac{20ml}{9kl-24mg}}$, при $k>24mg/9l$
\end{Answer}
\begin{Answer}{206}
$\omega_x = \sqrt{2k/m}$, $\omega_y=\sqrt{\frac{2k\Delta l}{m(l+\Delta l)}}$, $\Delta l = l/3$
\end{Answer}
\begin{Answer}{207}
$\omega = \sqrt{k_2/m_2}$
\end{Answer}
\begin{Answer}{208}
$x^2 = A^2\frac{\omega_0^4+4\beta^2\omega^2}{(\omega_0^2-\omega^2)^2+4\beta^2\omega^2}$, $\omega_0 = \sqrt{k/m}$, $\beta = b/2m$
\end{Answer}
\begin{Answer}{209}
$\omega^2=\frac{2\nu R S^2}{mV^2}T$
\end{Answer}
\begin{Answer}{210}
$T=2\pi\sqrt{\frac{ML}{p_0S + Mg}}$
\end{Answer}
\begin{Answer}{211}
$T=\pi \sqrt{2V/Sg}$
\end{Answer}
\begin{Answer}{212}
$T=2\pi \sqrt{H/g(1+\cos \alpha)}$
\end{Answer}
\begin{Answer}{213}
$T=2\pi\sqrt{L/g}$, $T=2(\pi+1)\sqrt{L/g}$
\end{Answer}
\begin{Answer}{214}
$\omega = \sqrt{2\alpha/mR^2-(\pi \eta R^2/m)^2}$
\end{Answer}
\begin{Answer}{215}
одно колебание
\end{Answer}
\begin{Answer}{216}
$T=2\pi \sqrt{\frac{L}{g(1+m/M)}}$
\end{Answer}
\begin{Answer}{217}
На расстоянии 5 см от вершины треугольника
\end{Answer}
\begin{Answer}{218}
$\alpha(t) = A \cos(\sqrt{\sqrt{a^2+g^2}/l}t) + \alpha_0$, $\alpha_0 = \arctan(a/g)$
\end{Answer}
\begin{Answer}{219}
1) $v=\sqrt{3gh}$; 2) $\nu = 0.25\sqrt{3g/h }$
\end{Answer}
\begin{Answer}{220}
$ml \varphi^{\prime \prime} + (mg-m \omega^2 A \cos \omega t) \varphi = 0$, $\omega^2 = T_0(S_1+S_2)/m S_1 S_2$
\end{Answer}
\begin{Answer}{221}
1) $A=qUL/d$; 2) $T=2\pi\sqrt{\frac{mLd}{2qU}}$; 3) $v_0 = \sqrt{\frac{qUL}{md }}$
\end{Answer}
\begin{Answer}{222}
\end{Answer}
\begin{Answer}{223}
$q_{\max} = \frac{CU}{2}\sqrt{2-\cos^2 \left( \sqrt{\frac{2}{LC}} \tau \right)}$
\end{Answer}
\begin{Answer}{224}
$I_{\max} = U \sqrt{\frac{C}{2L}}$, $T=2 \pi \sqrt{2LC}$
\end{Answer}
\begin{Answer}{225}
$q=\frac{q_0}{R}\sqrt{\frac{L}{C}}$
\end{Answer}
\begin{Answer}{226}
$\omega = 1/\sqrt{3LC}$
\end{Answer}
\begin{Answer}{227}
$\omega = 1/\sqrt{LC}$
\end{Answer}
\begin{Answer}{228}
$I(t) = \frac{\mathcal{E}_0}{R} \sin \omega t$
\end{Answer}
\begin{Answer}{229}
$L=R^2C$, $\tan \varphi = -\omega RC$
\end{Answer}
