\begin{sol}
1) В положении равновесия сумма сил тяжести перемычки и груза уравновешивается силой Ампера, действующей на перемычку вертикально вверх: $(m + M) g = I B L$. Сила тока в цепи: $I = \frac{\mathcal{E}}{R}$. Решая эти два уравнения относительно $m$ получим: $m = \frac{\mathcal{E} B L}{R g} - M$
2) Пусть установившаяся скорость перемычка равна $v$. В конуре возникает ЭДС индукции, направленная против ЭДС источника $\mathcal{E}$: $\mathcal{E}_i = v B L$. Установившаяся сила тока в контуре: $I_{уст} = \frac{\mathcal{E} - \mathcal{E}_i}{R} = \frac{\mathcal{E} - v B L}{R}$\\
При постоянной скорости перемычки сила Ампера будет уравновешиваться силой тяжести:
\begin{equation*}
mg = I_{уст} B L \qquad \rightarrow \qquad mg = \frac{\mathcal{E} - v B L}{R} B L
\end{equation} 
Масса была вычислена в предыдущем пункте, поэтому скорость равна: $v = \frac{m g R}{B^2 L^2}$
\end{sol}