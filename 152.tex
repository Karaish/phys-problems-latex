\begin{sol}
По теореме взаимности Грина: $Q_1\varphi_1+q\varphi+Q_2\varphi_2=0$, где $\varphi_1$, $\varphi_2$ и $\varphi$-потенциалы на поверхности малой, большой сферы и между ними соответственно, если поставить в геометрический центр сфер заряд $q`$($\varphi_1=\frac{q`}{a}$, $\varphi_2=\frac{q`}{b}$, $\varphi=\frac{q`}{r}$), $Q_1$, $Q_2$ и $q$-заряды на поверхности малой, большой сферы и между ними соответственно(при отсутствии других зарядов).\\ Так же система замкнута, поэтому действует закон сохранения заряда: $Q_1+Q_2+q=0$.\\
Решая эти уравнения относительно $Q_1$ и $Q_2$ можно получить: $Q_1=-q\frac{(b-r)a}{(b-a)r}$, $Q_2=-q\frac{(r-a)b}{(b-a)r}$
\end{sol}